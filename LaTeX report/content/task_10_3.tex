\documentclass[../main.tex]{subfiles}

\begin{document}
\section{Optimal Control of Pitch/Travel with Feedback (LQ)}
In this task we add feedback to the optimal controller that we developed in \cref{kap:Part2OptimalControlWithoutFeedback}

\subsection{LQ controller}
\textit{Briefly explain LQ controller. Especially, but not limited to, what is the role of the matrices Q and R? Justify your choice of weights.}

An LQ, or linear-quadratic, regulator is an optimal feedback controller that can be applied to a linear model $\Delta x=A\Delta x_i + B \Delta u_i$ with a quadratic cost function:
\begin{equation}
    J = \sum^\infty_{i=0} \Delta x_{i+1}^\top Q \Delta x_{i+1} + \Delta u_i^\top R \Delta u_i, \quad Q \ge0, \quad R > 0
\end{equation}
Here $ \Delta x = x - x^*$ and $\Delta u = u - u^*$ are deviations from the optimal trajectory.

The matrix $Q$ and the scalar $R$ are weights in the optimalization problem. The value of $Q$ determines how much deviation in the state value should be penalized, while the value of $R$ determines how much input-usage should be punished. This allows the designer to optimize the regulator to the specific implementation: A system where the input is cheap (like the helicopter used in this assignment) would have a relatively small value of $R$ compared to $Q$.

\subsection{Model Predictive Control}
\textit{Answer 10.3.1.3 here.}

\subsection{Experimental results}
\textit{Printouts of data from relevant experiments (plots).
Discussion and analysis of the results.
Answer 10.3.2.5 here.}

\subsection{MATLAB and Simulink}
\textit{Code and diagrams go here}
\end{document}